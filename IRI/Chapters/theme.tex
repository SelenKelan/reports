% chapitre de description du travail technique
\chapter{Objectives}

\label{theme}


\section{Objectives as presented}

The internship goals were, at first, defined as such:
\begin{itemize}
  \item Study of realistic models of autonomous robots
  \item Study of trajectory tracking strategies
  \item Implementation of trajectory tracking strategies in nominal case and robust case (in simulation)
  \item Implementation of Obstacle Avoidance
\end{itemize}

Although those goals seem well defined, those were given before my co-worker, Mr Bernardes had begun his internship.
Eventually, those goals changed in between.
When I arrived at the IRI, those goals had changed to such:

\begin{itemize}
  \item Study of realistic models of autonomous robots
  \item Study and Understanding of box particle filtering for state estimation
  \item Study of trajectory tracking strategies
  \item Implementation of trajectory tracking strategies
  \item Implementation of a control strategy to minimize the probability of shocks
\end{itemize}

\section{Evolutions and changes}

During the Internship, I had to adapt to all the changes necessary.
One of the biggest parts of my work was retro engineering the BPF code, and changing it from a once-over to a step-by-step system.
Unless this change was made, the program would have been simply unusable. It eventually took almost a third of my time.\\


\section{Issues and Challenges}

From what I could determine, Interval analysis, and vector fields were not something used usually at the UPC laboratory.
Such methods were greatly different to what they were used to, and we, Mr Bernardes and myself, brought a lot of new ideas and possibilities.
Eventually, the researchers' program was functionally better, in the sense that, even though their results had a slightly bigger error margin, their program was still much faster.\\
Still, those new ideas may yield new methods, new algorithms combining the best of both worlds.
