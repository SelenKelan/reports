% chapitre de description du travail technique
\chapter{Objectives}

\label{theme}


\section{Initial Objectives}

The internship objectives were, initially, defined as follows:
\begin{itemize}
  \item Study of realistic models of autonomous robots
  \item Study of trajectory tracking strategies
  \item Implementation of trajectory tracking strategies in nominal case and robust case (in simulation)
  \item Implementation of Obstacle Avoidance
\end{itemize}

Although those goals seem well defined, there were given before my co-worker, Mr Bernardes had begun his internship.
Eventually, those goals changed in between.
At the beginning of my stay at the IRI, those goals had changed to:

\begin{itemize}
  \item Study of realistic models of autonomous robots
  \item Study and understanding of box particle filtering for state estimation
  \item Study of trajectory tracking strategies
  \item Implementation of trajectory tracking strategies
  \item Implementation of a control strategy to minimize the probability of shocks
\end{itemize}

\section{Evolutions and Changes}

It has been necessary to adapt to all these changes in the course of the internship.
 One of the biggest tasks was retroengineering of the BPF code, and changing it from a once-over to a step-by-step system.
Without this change, the program would have been simply unusable. It eventually took almost a third of the allotted time.\\

\section{Issues and Challenges}

To my knowledge, interval analysis and vector fields were not used commonly at the UPC laboratory. 
Such methods were greatly different from what was in use, and we could, Mr Bernardes and myself, bring some new ideas and possibilities.
Finally, the existing program was functionally better, in the sense that it was still much faster, even though the results showed a slightly bigger error margin.
Still, those new ideas may yield new methods and new algorithms combining best of both worlds.
