% chapitre de description du travail technique
\chapter{Contributions to the professionnal project}

\label{apport}


\section{Skills Learnt}
Among what I learnt during this internship, most of the things were linked directly or indirectly to technical knowledge.

The technical skills I learnt were as such:

\begin{itemize}
  \item Knowledge of state estimation techniques for autonomous robots using box particle filtering
  \item Knowledge of trajectory tracking and avoidance strategies
  \item Implementation in the Matlab environement of nominal and robust tracking error methods
  \item Implementation in the Matlab environement of obstacle avoidance methods
\end{itemize}

But, of course, this can't be the only contribution. A great part of what I learnt was interpersonal skill in an international environement.
Although it was far from my first trip to another country, it was the first time I had to work in another country than France.
And, as in many of the countries with a strong latin influence, professional and private sphere have blurred boundaries.

This is even more highlighted in laboratory work, were there are no work schedules.\\

As a result, a good part of my work was done in a bar, at home, or anywhere with air conditioning.
Discussions with coworkers were free-flowing, and no hierarchical boundaries were felt during these times of exchange.\\

This motus operandi allowed me to greatly improve my conversational and interpersonnal skills.

\section{Public research and Private Business}

I already had a pretty clear view of what I wanted to do after the engineering school, but this experience precised it even more.

I have a preference for Research and Developpement, but sadly, the free-flowing, even laid back management of a public laboratory is not something I like.
Through my two internships, one at ENSTA Bretagne, and the other at the UPC, I can now say without a doubt that something like this is not made for me.\\

My next internship will be made in a private structure, to understand the differences, and maybe validate my choice.
