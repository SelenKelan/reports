%contexte

\chapter{Mission of the research center}
\label{context}

\section{The Polytechnic University of Catalonia}

The Polytechnic University of Catalonia, named in Catalan Universitat Politècnica de Catalunya, currently referred to as BarcelonaTech and commonly named just as UPC, is the largest engineering university in Catalonia, Spain.
It also offers programs in other disciplines such as mathematics and architecture.

UPC's objectives are based on internationalization, as it is one of Europe's technical universities with the most international PhD students and the university with the largest share of international master's degree students.
UPC is a university aiming at achieving the highest degree of engineering/technical excellence and has bilateral agreements with several top-ranked European universities.


UPC is a member of the Top Industrial Managers for Europe network, which allows for student exchanges between leading European engineering schools.
It is also a member of several university federations, including the Conference of European Schools for Advanced Engineering Education and Research (CESAER) and UNITECH.

The university was founded in March 1971 as the Universitat Politècnica de Barcelona through the merger of engineering and architecture schools founded during the 19th century.
As of 2007 it has 25 schools in Catalonia located in the cities of Barcelona, Castelldefels, Manresa, Sant Cugat del Vallès, Terrassa, Igualada, Vilanova i la Geltrú and Mataró.
UPC has about 30,000 students and 2,500 professors and researchers

\section{Institut de Robòtica i Informàtica Industrial}



The Institut de Robòtica i Informàtica Industrial is a Joint Research Center of the Spanish Council for Scientific Research (CSIC) and the Technical University of Catalonia (UPC).

 The Institute has three main objectives: to promote fundamental research in Robotics and Applied Informatics, to cooperate with the community in industrial technological projects, and to offer scientific education through graduate courses.
The Institute's research activities are organized in four research lines.

Three of them tackle various aspects of robotics research, including indoor and outdoor human-centered human-safe robotics systems, and the design and construction of novel parallel mechanisms.
Efforts in the fourth line are aimed at research on energy efficiency, in fuel cells research, and on the management of energy systems.
